\documentclass[letterpaper,12pt,onecolumn]{IEEEtran}
%draftclsnofoot

\usepackage{hyperref}
\usepackage{geometry}

\geometry{margin=1.0 in}
\linespread{1.5}
\setlength{\parskip}{1em}
\pagenumbering{gobble}

\begin{document}

\noindent
Rhea Mae V. Edwards\\
Fall 2017 | PAC 152\\ 
Tuesdays and Thursdays 9AM

\section*{Salsa Dancing the Stress Away}

% 300 to 500 words
% Observations and experiences as a dancer
% Most importantly, what did I learn
% Include date and location of the dance

It was the night of a very stressful day filled with frustration and anxiety, but my experience at Impulse Bar and Grill was going to provide the relief I needed and more. The night started off with three one-dollar tacos and a club filled with new and familiar faces. With friends' birthdays just days away after that night of November 7th, 2017, starting off the celebrations with salsa dancing, seemed like a great event to bring on the smiles.

Not having done salsa dancing in a filled room with other salsa dancers since I started taking Salsa 1, that night was definitely a new experience to go through. I observed the ways other couples danced, and felt the differences in the environment around me compared to the usual classroom setting in the Women's Building room 116. Overall, I experienced and learned more that what I was expecting.

First off, a few observations that I noticed were the variations of the beats in the music played, the different styles of moves other couples preformed around me, and the constant change of available space as a couple to dance on the dance floor around us. 

In regards to the beats of each song that was played, they seemed different in a way than what I am used to. I cannot fully explain the difference, rather than it was. Most of the music felt like I was moving to the rhythm more than the typical and basic step patterns; you just go with it in a sense. Some songs did feel faster, while others were slower. I realized most of the time, you just have to move in a way that is proper and just feels right for both you and your partner.

Another observation that I really did not think much about until I saw it, was the ranging styles of salsa dancing other couples were doing around me. Some couples' dance moves were more smooth and flowing more from move to move, while others were more quick with spins from left to right, while all dancing to the same song. Viewing these varieties in styles made me realize there is truly no right way to dance to any salsa song, and well, to any genre of song for that matter.

One of the more obvious observations that most people would notice, is the actual space to dance. Thankfully, during the night I went, it was not too crowded, but there was a good amount of couples dancing on the floor from time to time. The amount of couples on the dance floor, directly played a part with the range of movement you and your partner had to move. I finally got to experience those changes within one night of dancing, and was quickly able to compare the varying experiences, which I found very insightful and fantastic in actually being able to go through such a dynamic environment.

In addition, my dance partner for most of the night taught me some new dance moves as we also practiced moves I learned from class previously. Moves that are not that easy to describe simply into just words, but definitely ones I personally enjoyed and will hopefully remember for the future. 

In the end, I danced more than what my mind and body could handle. The day I had was exhausting, and I was able to gather just enough energy that exceeded my expectations of how much fun I actually did have that night. I was able to learn new moves from my partner, have a feel about how I should react and be aware of my surroundings when dancing, and all in all, creating new memories and having a lot of fun in doing so. Later that night, as I finally laid in bed to relax, the thought came into mind, "would I do that all again?", absolutely.

\end{document}

